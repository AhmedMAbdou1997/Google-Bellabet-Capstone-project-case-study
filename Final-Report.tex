% Options for packages loaded elsewhere
\PassOptionsToPackage{unicode}{hyperref}
\PassOptionsToPackage{hyphens}{url}
%
\documentclass[
]{article}
\usepackage{amsmath,amssymb}
\usepackage{iftex}
\ifPDFTeX
  \usepackage[T1]{fontenc}
  \usepackage[utf8]{inputenc}
  \usepackage{textcomp} % provide euro and other symbols
\else % if luatex or xetex
  \usepackage{unicode-math} % this also loads fontspec
  \defaultfontfeatures{Scale=MatchLowercase}
  \defaultfontfeatures[\rmfamily]{Ligatures=TeX,Scale=1}
\fi
\usepackage{lmodern}
\ifPDFTeX\else
  % xetex/luatex font selection
\fi
% Use upquote if available, for straight quotes in verbatim environments
\IfFileExists{upquote.sty}{\usepackage{upquote}}{}
\IfFileExists{microtype.sty}{% use microtype if available
  \usepackage[]{microtype}
  \UseMicrotypeSet[protrusion]{basicmath} % disable protrusion for tt fonts
}{}
\makeatletter
\@ifundefined{KOMAClassName}{% if non-KOMA class
  \IfFileExists{parskip.sty}{%
    \usepackage{parskip}
  }{% else
    \setlength{\parindent}{0pt}
    \setlength{\parskip}{6pt plus 2pt minus 1pt}}
}{% if KOMA class
  \KOMAoptions{parskip=half}}
\makeatother
\usepackage{xcolor}
\usepackage[margin=1in]{geometry}
\usepackage{color}
\usepackage{fancyvrb}
\newcommand{\VerbBar}{|}
\newcommand{\VERB}{\Verb[commandchars=\\\{\}]}
\DefineVerbatimEnvironment{Highlighting}{Verbatim}{commandchars=\\\{\}}
% Add ',fontsize=\small' for more characters per line
\usepackage{framed}
\definecolor{shadecolor}{RGB}{248,248,248}
\newenvironment{Shaded}{\begin{snugshade}}{\end{snugshade}}
\newcommand{\AlertTok}[1]{\textcolor[rgb]{0.94,0.16,0.16}{#1}}
\newcommand{\AnnotationTok}[1]{\textcolor[rgb]{0.56,0.35,0.01}{\textbf{\textit{#1}}}}
\newcommand{\AttributeTok}[1]{\textcolor[rgb]{0.13,0.29,0.53}{#1}}
\newcommand{\BaseNTok}[1]{\textcolor[rgb]{0.00,0.00,0.81}{#1}}
\newcommand{\BuiltInTok}[1]{#1}
\newcommand{\CharTok}[1]{\textcolor[rgb]{0.31,0.60,0.02}{#1}}
\newcommand{\CommentTok}[1]{\textcolor[rgb]{0.56,0.35,0.01}{\textit{#1}}}
\newcommand{\CommentVarTok}[1]{\textcolor[rgb]{0.56,0.35,0.01}{\textbf{\textit{#1}}}}
\newcommand{\ConstantTok}[1]{\textcolor[rgb]{0.56,0.35,0.01}{#1}}
\newcommand{\ControlFlowTok}[1]{\textcolor[rgb]{0.13,0.29,0.53}{\textbf{#1}}}
\newcommand{\DataTypeTok}[1]{\textcolor[rgb]{0.13,0.29,0.53}{#1}}
\newcommand{\DecValTok}[1]{\textcolor[rgb]{0.00,0.00,0.81}{#1}}
\newcommand{\DocumentationTok}[1]{\textcolor[rgb]{0.56,0.35,0.01}{\textbf{\textit{#1}}}}
\newcommand{\ErrorTok}[1]{\textcolor[rgb]{0.64,0.00,0.00}{\textbf{#1}}}
\newcommand{\ExtensionTok}[1]{#1}
\newcommand{\FloatTok}[1]{\textcolor[rgb]{0.00,0.00,0.81}{#1}}
\newcommand{\FunctionTok}[1]{\textcolor[rgb]{0.13,0.29,0.53}{\textbf{#1}}}
\newcommand{\ImportTok}[1]{#1}
\newcommand{\InformationTok}[1]{\textcolor[rgb]{0.56,0.35,0.01}{\textbf{\textit{#1}}}}
\newcommand{\KeywordTok}[1]{\textcolor[rgb]{0.13,0.29,0.53}{\textbf{#1}}}
\newcommand{\NormalTok}[1]{#1}
\newcommand{\OperatorTok}[1]{\textcolor[rgb]{0.81,0.36,0.00}{\textbf{#1}}}
\newcommand{\OtherTok}[1]{\textcolor[rgb]{0.56,0.35,0.01}{#1}}
\newcommand{\PreprocessorTok}[1]{\textcolor[rgb]{0.56,0.35,0.01}{\textit{#1}}}
\newcommand{\RegionMarkerTok}[1]{#1}
\newcommand{\SpecialCharTok}[1]{\textcolor[rgb]{0.81,0.36,0.00}{\textbf{#1}}}
\newcommand{\SpecialStringTok}[1]{\textcolor[rgb]{0.31,0.60,0.02}{#1}}
\newcommand{\StringTok}[1]{\textcolor[rgb]{0.31,0.60,0.02}{#1}}
\newcommand{\VariableTok}[1]{\textcolor[rgb]{0.00,0.00,0.00}{#1}}
\newcommand{\VerbatimStringTok}[1]{\textcolor[rgb]{0.31,0.60,0.02}{#1}}
\newcommand{\WarningTok}[1]{\textcolor[rgb]{0.56,0.35,0.01}{\textbf{\textit{#1}}}}
\usepackage{graphicx}
\makeatletter
\def\maxwidth{\ifdim\Gin@nat@width>\linewidth\linewidth\else\Gin@nat@width\fi}
\def\maxheight{\ifdim\Gin@nat@height>\textheight\textheight\else\Gin@nat@height\fi}
\makeatother
% Scale images if necessary, so that they will not overflow the page
% margins by default, and it is still possible to overwrite the defaults
% using explicit options in \includegraphics[width, height, ...]{}
\setkeys{Gin}{width=\maxwidth,height=\maxheight,keepaspectratio}
% Set default figure placement to htbp
\makeatletter
\def\fps@figure{htbp}
\makeatother
\setlength{\emergencystretch}{3em} % prevent overfull lines
\providecommand{\tightlist}{%
  \setlength{\itemsep}{0pt}\setlength{\parskip}{0pt}}
\setcounter{secnumdepth}{-\maxdimen} % remove section numbering
\ifLuaTeX
  \usepackage{selnolig}  % disable illegal ligatures
\fi
\usepackage{bookmark}
\IfFileExists{xurl.sty}{\usepackage{xurl}}{} % add URL line breaks if available
\urlstyle{same}
\hypersetup{
  pdftitle={Google Final Case Study Project Case Study: Fitness Tracking App - Calorie Related Analysis},
  pdfauthor={Ahmed M. Abdo},
  hidelinks,
  pdfcreator={LaTeX via pandoc}}

\title{Google Final Case Study Project Case Study: Fitness Tracking App
- Calorie Related Analysis}
\author{Ahmed M. Abdo}
\date{2025-02-05}

\begin{document}
\maketitle

\begin{Shaded}
\begin{Highlighting}[]
\NormalTok{knitr}\SpecialCharTok{::}\NormalTok{opts\_chunk}\SpecialCharTok{$}\FunctionTok{set}\NormalTok{(}\AttributeTok{include =} \ConstantTok{FALSE}\NormalTok{)}
\CommentTok{\#Libraries Loading}
\ControlFlowTok{if}\NormalTok{(}\SpecialCharTok{!}\FunctionTok{require}\NormalTok{(tidyverse))\{}\FunctionTok{install.packages}\NormalTok{(}\StringTok{"tidyverse"}\NormalTok{)\}}
\end{Highlighting}
\end{Shaded}

\begin{verbatim}
## Loading required package: tidyverse
\end{verbatim}

\begin{verbatim}
## -- Attaching core tidyverse packages ------------------------ tidyverse 2.0.0 --
## v dplyr     1.1.4     v readr     2.1.5
## v forcats   1.0.0     v stringr   1.5.1
## v ggplot2   3.5.1     v tibble    3.2.1
## v lubridate 1.9.4     v tidyr     1.3.1
## v purrr     1.0.2     
## -- Conflicts ------------------------------------------ tidyverse_conflicts() --
## x dplyr::filter() masks stats::filter()
## x dplyr::lag()    masks stats::lag()
## i Use the conflicted package (<http://conflicted.r-lib.org/>) to force all conflicts to become errors
\end{verbatim}

\begin{Shaded}
\begin{Highlighting}[]
\ControlFlowTok{if}\NormalTok{(}\SpecialCharTok{!}\FunctionTok{require}\NormalTok{(gridExtra))\{}\FunctionTok{install.packages}\NormalTok{(}\StringTok{"gridExtra"}\NormalTok{)\}}
\end{Highlighting}
\end{Shaded}

\begin{verbatim}
## Loading required package: gridExtra
## 
## Attaching package: 'gridExtra'
## 
## The following object is masked from 'package:dplyr':
## 
##     combine
\end{verbatim}

\begin{Shaded}
\begin{Highlighting}[]
\FunctionTok{library}\NormalTok{(tidyverse)}
\FunctionTok{library}\NormalTok{(readr)}
\FunctionTok{library}\NormalTok{(dplyr)}
\FunctionTok{library}\NormalTok{(ggplot2)}
\FunctionTok{library}\NormalTok{(gridExtra)}
\end{Highlighting}
\end{Shaded}

\section{1. Problem Definition}\label{problem-definition}

This case study investigates the relationship between calorie burn and
various parameters collected by a fitness tracking app. The primary goal
is to understand which factors are most influential in calorie
expenditure and demonstrate the importance of calorie tracking for
overall health management. We will explore how different activities,
duration, intensity, steps, sleep, and weight contribute to calorie
burn. Ultimately, this analysis aims to provide actionable insights for
users to optimize their fitness routines and achieve their health goals.

\section{2. Data Collection}\label{data-collection}

The data for this analysis comes from a publicly available Fitbit
dataset. It includes various metrics related to daily activity, hourly
activity, sleep, and weight. The following datasets were used:

\begin{itemize}
\item
  dailyActivity\_merged.csv: Daily summary of activity.
\item
  dailyCalories\_merged.csv: Daily calorie estimates.
\item
  dailyIntensities\_merged.csv: Daily intensity levels.
\item
  dailySteps\_merged.csv: Daily step counts.
\item
  hourlyCalories\_merged.csv: Hourly calorie estimates.
\item
  hourlyIntensities\_merged.csv: Hourly intensity levels.
\item
  hourlySteps\_merged.csv: Hourly step counts.
\item
  sleepDay\_merged.csv: Daily sleep records.
\item
  heartrate\_seconds\_merged.csv: Second-by-second heart rate data (not
  used in this analysis due to complexity).
\item
  weightLogInfo\_merged.csv: Weight logs.
\item
  minuteCaloriesNarrow\_merged.csv, minuteIntensitiesNarrow\_merged.csv,
  minuteMETsNarrow\_merged.csv, minuteStepsNarrow\_merged.csv,
  minuteSleep\_merged.csv:\\
  Minute-level data (not used in this initial analysis due to
  aggregation requirements).
\end{itemize}

\section{3. Data Cleaning and
Preprocessing}\label{data-cleaning-and-preprocessing}

\subsubsection{The Following Code Loads Data That we gonna work with
Starting With the entire
dataset.}\label{the-following-code-loads-data-that-we-gonna-work-with-starting-with-the-entire-dataset.}

\begin{Shaded}
\begin{Highlighting}[]
\CommentTok{\#Data loading in Variables}
\NormalTok{activity }\OtherTok{\textless{}{-}} \FunctionTok{read.csv}\NormalTok{(}\StringTok{"dailyActivity\_merged.csv"}\NormalTok{)}
\NormalTok{calories }\OtherTok{\textless{}{-}} \FunctionTok{read.csv}\NormalTok{(}\StringTok{"dailyCalories\_merged.csv"}\NormalTok{)}
\NormalTok{intensities }\OtherTok{\textless{}{-}} \FunctionTok{read.csv}\NormalTok{(}\StringTok{"dailyIntensities\_merged.csv"}\NormalTok{)}
\NormalTok{steps }\OtherTok{\textless{}{-}} \FunctionTok{read.csv}\NormalTok{(}\StringTok{"dailySteps\_merged.csv"}\NormalTok{)}
\NormalTok{hcalories }\OtherTok{\textless{}{-}} \FunctionTok{read.csv}\NormalTok{(}\StringTok{"hourlyCalories\_merged.csv"}\NormalTok{)}
\NormalTok{hintensities }\OtherTok{\textless{}{-}} \FunctionTok{read.csv}\NormalTok{(}\StringTok{"hourlyIntensities\_merged.csv"}\NormalTok{)}
\NormalTok{hsteps }\OtherTok{\textless{}{-}} \FunctionTok{read.csv}\NormalTok{(}\StringTok{"hourlySteps\_merged.csv"}\NormalTok{)}
\NormalTok{dsleep }\OtherTok{\textless{}{-}} \FunctionTok{read.csv}\NormalTok{(}\StringTok{"sleepDay\_merged.csv"}\NormalTok{)}
\NormalTok{heartrate }\OtherTok{\textless{}{-}} \FunctionTok{read.csv}\NormalTok{(}\StringTok{"heartrate\_seconds\_merged.csv"}\NormalTok{)}
\NormalTok{msleep }\OtherTok{\textless{}{-}} \FunctionTok{read.csv}\NormalTok{(}\StringTok{"minuteSleep\_merged.csv"}\NormalTok{)}
\NormalTok{weight }\OtherTok{\textless{}{-}} \FunctionTok{read.csv}\NormalTok{(}\StringTok{"weightLogInfo\_merged.csv"}\NormalTok{)}
\NormalTok{mcalories }\OtherTok{\textless{}{-}} \FunctionTok{read\_csv}\NormalTok{(}\StringTok{"minuteCaloriesNarrow\_merged.csv"}\NormalTok{)}
\end{Highlighting}
\end{Shaded}

\begin{verbatim}
## Rows: 1325580 Columns: 3
## -- Column specification --------------------------------------------------------
## Delimiter: ","
## chr (1): ActivityMinute
## dbl (2): Id, Calories
## 
## i Use `spec()` to retrieve the full column specification for this data.
## i Specify the column types or set `show_col_types = FALSE` to quiet this message.
\end{verbatim}

\begin{Shaded}
\begin{Highlighting}[]
\NormalTok{mintensities }\OtherTok{\textless{}{-}} \FunctionTok{read\_csv}\NormalTok{(}\StringTok{"minuteIntensitiesNarrow\_merged.csv"}\NormalTok{)}
\end{Highlighting}
\end{Shaded}

\begin{verbatim}
## Rows: 1325580 Columns: 3
## -- Column specification --------------------------------------------------------
## Delimiter: ","
## chr (1): ActivityMinute
## dbl (2): Id, Intensity
## 
## i Use `spec()` to retrieve the full column specification for this data.
## i Specify the column types or set `show_col_types = FALSE` to quiet this message.
\end{verbatim}

\begin{Shaded}
\begin{Highlighting}[]
\NormalTok{mmets }\OtherTok{\textless{}{-}} \FunctionTok{read\_csv}\NormalTok{(}\StringTok{"minuteMETsNarrow\_merged.csv"}\NormalTok{)}
\end{Highlighting}
\end{Shaded}

\begin{verbatim}
## Rows: 1325580 Columns: 3
## -- Column specification --------------------------------------------------------
## Delimiter: ","
## chr (1): ActivityMinute
## dbl (2): Id, METs
## 
## i Use `spec()` to retrieve the full column specification for this data.
## i Specify the column types or set `show_col_types = FALSE` to quiet this message.
\end{verbatim}

\begin{Shaded}
\begin{Highlighting}[]
\NormalTok{msteps }\OtherTok{\textless{}{-}} \FunctionTok{read\_csv}\NormalTok{(}\StringTok{"minuteStepsNarrow\_merged.csv"}\NormalTok{)}
\end{Highlighting}
\end{Shaded}

\begin{verbatim}
## Rows: 1325580 Columns: 3
## -- Column specification --------------------------------------------------------
## Delimiter: ","
## chr (1): ActivityMinute
## dbl (2): Id, Steps
## 
## i Use `spec()` to retrieve the full column specification for this data.
## i Specify the column types or set `show_col_types = FALSE` to quiet this message.
\end{verbatim}

\#\#\#Checking Columns for Data Integrity and Check Our needs of the
Data and understanding what each dataframe have to offer.

\begin{Shaded}
\begin{Highlighting}[]
\CommentTok{\#Checking Data Validation }
\FunctionTok{colnames}\NormalTok{(activity)}
\end{Highlighting}
\end{Shaded}

\begin{verbatim}
##  [1] "Id"                       "ActivityDate"            
##  [3] "TotalSteps"               "TotalDistance"           
##  [5] "TrackerDistance"          "LoggedActivitiesDistance"
##  [7] "VeryActiveDistance"       "ModeratelyActiveDistance"
##  [9] "LightActiveDistance"      "SedentaryActiveDistance" 
## [11] "VeryActiveMinutes"        "FairlyActiveMinutes"     
## [13] "LightlyActiveMinutes"     "SedentaryMinutes"        
## [15] "Calories"
\end{verbatim}

\begin{Shaded}
\begin{Highlighting}[]
\FunctionTok{colnames}\NormalTok{(calories)}
\end{Highlighting}
\end{Shaded}

\begin{verbatim}
## [1] "Id"          "ActivityDay" "Calories"
\end{verbatim}

\begin{Shaded}
\begin{Highlighting}[]
\FunctionTok{colnames}\NormalTok{(intensities)}
\end{Highlighting}
\end{Shaded}

\begin{verbatim}
##  [1] "Id"                       "ActivityDay"             
##  [3] "SedentaryMinutes"         "LightlyActiveMinutes"    
##  [5] "FairlyActiveMinutes"      "VeryActiveMinutes"       
##  [7] "SedentaryActiveDistance"  "LightActiveDistance"     
##  [9] "ModeratelyActiveDistance" "VeryActiveDistance"
\end{verbatim}

\begin{Shaded}
\begin{Highlighting}[]
\FunctionTok{colnames}\NormalTok{(steps)}
\end{Highlighting}
\end{Shaded}

\begin{verbatim}
## [1] "Id"          "ActivityDay" "StepTotal"
\end{verbatim}

\begin{Shaded}
\begin{Highlighting}[]
\FunctionTok{colnames}\NormalTok{(heartrate)}
\end{Highlighting}
\end{Shaded}

\begin{verbatim}
## [1] "Id"    "Time"  "Value"
\end{verbatim}

\begin{Shaded}
\begin{Highlighting}[]
\FunctionTok{colnames}\NormalTok{(hcalories)}
\end{Highlighting}
\end{Shaded}

\begin{verbatim}
## [1] "Id"           "ActivityHour" "Calories"
\end{verbatim}

\begin{Shaded}
\begin{Highlighting}[]
\FunctionTok{colnames}\NormalTok{(hintensities)}
\end{Highlighting}
\end{Shaded}

\begin{verbatim}
## [1] "Id"               "ActivityHour"     "TotalIntensity"   "AverageIntensity"
\end{verbatim}

\begin{Shaded}
\begin{Highlighting}[]
\FunctionTok{colnames}\NormalTok{(hsteps)}
\end{Highlighting}
\end{Shaded}

\begin{verbatim}
## [1] "Id"           "ActivityHour" "StepTotal"
\end{verbatim}

\begin{Shaded}
\begin{Highlighting}[]
\FunctionTok{colnames}\NormalTok{(mcalories)}
\end{Highlighting}
\end{Shaded}

\begin{verbatim}
## [1] "Id"             "ActivityMinute" "Calories"
\end{verbatim}

\begin{Shaded}
\begin{Highlighting}[]
\FunctionTok{colnames}\NormalTok{(mintensities)}
\end{Highlighting}
\end{Shaded}

\begin{verbatim}
## [1] "Id"             "ActivityMinute" "Intensity"
\end{verbatim}

\begin{Shaded}
\begin{Highlighting}[]
\FunctionTok{colnames}\NormalTok{(mmets)}
\end{Highlighting}
\end{Shaded}

\begin{verbatim}
## [1] "Id"             "ActivityMinute" "METs"
\end{verbatim}

\begin{Shaded}
\begin{Highlighting}[]
\FunctionTok{colnames}\NormalTok{(msleep)}
\end{Highlighting}
\end{Shaded}

\begin{verbatim}
## [1] "Id"    "date"  "value" "logId"
\end{verbatim}

\begin{Shaded}
\begin{Highlighting}[]
\FunctionTok{colnames}\NormalTok{(msteps)}
\end{Highlighting}
\end{Shaded}

\begin{verbatim}
## [1] "Id"             "ActivityMinute" "Steps"
\end{verbatim}

\begin{Shaded}
\begin{Highlighting}[]
\FunctionTok{colnames}\NormalTok{(weight)}
\end{Highlighting}
\end{Shaded}

\begin{verbatim}
## [1] "Id"             "Date"           "WeightKg"       "WeightPounds"  
## [5] "Fat"            "BMI"            "IsManualReport" "LogId"
\end{verbatim}

\begin{Shaded}
\begin{Highlighting}[]
\FunctionTok{colnames}\NormalTok{(dsleep)}
\end{Highlighting}
\end{Shaded}

\begin{verbatim}
## [1] "Id"                 "SleepDay"           "TotalSleepRecords" 
## [4] "TotalMinutesAsleep" "TotalTimeInBed"
\end{verbatim}

\begin{Shaded}
\begin{Highlighting}[]
\FunctionTok{length}\NormalTok{(}\FunctionTok{unique}\NormalTok{(activity}\SpecialCharTok{$}\NormalTok{Id))}
\end{Highlighting}
\end{Shaded}

\begin{verbatim}
## [1] 33
\end{verbatim}

\begin{Shaded}
\begin{Highlighting}[]
\FunctionTok{length}\NormalTok{(}\FunctionTok{unique}\NormalTok{(heartrate}\SpecialCharTok{$}\NormalTok{Id))}
\end{Highlighting}
\end{Shaded}

\begin{verbatim}
## [1] 14
\end{verbatim}

\begin{Shaded}
\begin{Highlighting}[]
\FunctionTok{length}\NormalTok{(}\FunctionTok{unique}\NormalTok{(hcalories}\SpecialCharTok{$}\NormalTok{Id))}
\end{Highlighting}
\end{Shaded}

\begin{verbatim}
## [1] 33
\end{verbatim}

\begin{Shaded}
\begin{Highlighting}[]
\FunctionTok{length}\NormalTok{(}\FunctionTok{unique}\NormalTok{(hintensities}\SpecialCharTok{$}\NormalTok{Id))}
\end{Highlighting}
\end{Shaded}

\begin{verbatim}
## [1] 33
\end{verbatim}

\begin{Shaded}
\begin{Highlighting}[]
\FunctionTok{length}\NormalTok{(}\FunctionTok{unique}\NormalTok{(hsteps}\SpecialCharTok{$}\NormalTok{Id))}
\end{Highlighting}
\end{Shaded}

\begin{verbatim}
## [1] 33
\end{verbatim}

\begin{Shaded}
\begin{Highlighting}[]
\FunctionTok{length}\NormalTok{(}\FunctionTok{unique}\NormalTok{(mcalories}\SpecialCharTok{$}\NormalTok{Id))}
\end{Highlighting}
\end{Shaded}

\begin{verbatim}
## [1] 33
\end{verbatim}

\begin{Shaded}
\begin{Highlighting}[]
\FunctionTok{length}\NormalTok{(}\FunctionTok{unique}\NormalTok{(mintensities}\SpecialCharTok{$}\NormalTok{Id))}
\end{Highlighting}
\end{Shaded}

\begin{verbatim}
## [1] 33
\end{verbatim}

\begin{Shaded}
\begin{Highlighting}[]
\FunctionTok{length}\NormalTok{(}\FunctionTok{unique}\NormalTok{(mmets}\SpecialCharTok{$}\NormalTok{Id))}
\end{Highlighting}
\end{Shaded}

\begin{verbatim}
## [1] 33
\end{verbatim}

\begin{Shaded}
\begin{Highlighting}[]
\FunctionTok{length}\NormalTok{(}\FunctionTok{unique}\NormalTok{(msleep}\SpecialCharTok{$}\NormalTok{Id))}
\end{Highlighting}
\end{Shaded}

\begin{verbatim}
## [1] 24
\end{verbatim}

\begin{Shaded}
\begin{Highlighting}[]
\FunctionTok{length}\NormalTok{(}\FunctionTok{unique}\NormalTok{(msteps}\SpecialCharTok{$}\NormalTok{Id))}
\end{Highlighting}
\end{Shaded}

\begin{verbatim}
## [1] 33
\end{verbatim}

\begin{Shaded}
\begin{Highlighting}[]
\FunctionTok{length}\NormalTok{(}\FunctionTok{unique}\NormalTok{(weight}\SpecialCharTok{$}\NormalTok{Id))}
\end{Highlighting}
\end{Shaded}

\begin{verbatim}
## [1] 8
\end{verbatim}

\begin{Shaded}
\begin{Highlighting}[]
\FunctionTok{length}\NormalTok{(}\FunctionTok{unique}\NormalTok{(calories}\SpecialCharTok{$}\NormalTok{Id))}
\end{Highlighting}
\end{Shaded}

\begin{verbatim}
## [1] 33
\end{verbatim}

\begin{Shaded}
\begin{Highlighting}[]
\FunctionTok{length}\NormalTok{(}\FunctionTok{unique}\NormalTok{(intensities}\SpecialCharTok{$}\NormalTok{Id))}
\end{Highlighting}
\end{Shaded}

\begin{verbatim}
## [1] 33
\end{verbatim}

\begin{Shaded}
\begin{Highlighting}[]
\FunctionTok{length}\NormalTok{(}\FunctionTok{unique}\NormalTok{(steps}\SpecialCharTok{$}\NormalTok{Id))}
\end{Highlighting}
\end{Shaded}

\begin{verbatim}
## [1] 33
\end{verbatim}

\begin{Shaded}
\begin{Highlighting}[]
\FunctionTok{length}\NormalTok{(}\FunctionTok{unique}\NormalTok{(dsleep}\SpecialCharTok{$}\NormalTok{Id))}
\end{Highlighting}
\end{Shaded}

\begin{verbatim}
## [1] 24
\end{verbatim}

\subsubsection{Finally Undstanding our Analysis Requirement and Begin to
Work
With}\label{finally-undstanding-our-analysis-requirement-and-begin-to-work-with}

\begin{Shaded}
\begin{Highlighting}[]
\CommentTok{\# Choosing Which Tables to work With}

\FunctionTok{head}\NormalTok{(activity)}
\end{Highlighting}
\end{Shaded}

\begin{verbatim}
##           Id ActivityDate TotalSteps TotalDistance TrackerDistance
## 1 1503960366    4/12/2016      13162          8.50            8.50
## 2 1503960366    4/13/2016      10735          6.97            6.97
## 3 1503960366    4/14/2016      10460          6.74            6.74
## 4 1503960366    4/15/2016       9762          6.28            6.28
## 5 1503960366    4/16/2016      12669          8.16            8.16
## 6 1503960366    4/17/2016       9705          6.48            6.48
##   LoggedActivitiesDistance VeryActiveDistance ModeratelyActiveDistance
## 1                        0               1.88                     0.55
## 2                        0               1.57                     0.69
## 3                        0               2.44                     0.40
## 4                        0               2.14                     1.26
## 5                        0               2.71                     0.41
## 6                        0               3.19                     0.78
##   LightActiveDistance SedentaryActiveDistance VeryActiveMinutes
## 1                6.06                       0                25
## 2                4.71                       0                21
## 3                3.91                       0                30
## 4                2.83                       0                29
## 5                5.04                       0                36
## 6                2.51                       0                38
##   FairlyActiveMinutes LightlyActiveMinutes SedentaryMinutes Calories
## 1                  13                  328              728     1985
## 2                  19                  217              776     1797
## 3                  11                  181             1218     1776
## 4                  34                  209              726     1745
## 5                  10                  221              773     1863
## 6                  20                  164              539     1728
\end{verbatim}

\begin{Shaded}
\begin{Highlighting}[]
\FunctionTok{head}\NormalTok{(calories)}
\end{Highlighting}
\end{Shaded}

\begin{verbatim}
##           Id ActivityDay Calories
## 1 1503960366   4/12/2016     1985
## 2 1503960366   4/13/2016     1797
## 3 1503960366   4/14/2016     1776
## 4 1503960366   4/15/2016     1745
## 5 1503960366   4/16/2016     1863
## 6 1503960366   4/17/2016     1728
\end{verbatim}

\begin{Shaded}
\begin{Highlighting}[]
\FunctionTok{head}\NormalTok{(dsleep)}
\end{Highlighting}
\end{Shaded}

\begin{verbatim}
##           Id              SleepDay TotalSleepRecords TotalMinutesAsleep
## 1 1503960366 4/12/2016 12:00:00 AM                 1                327
## 2 1503960366 4/13/2016 12:00:00 AM                 2                384
## 3 1503960366 4/15/2016 12:00:00 AM                 1                412
## 4 1503960366 4/16/2016 12:00:00 AM                 2                340
## 5 1503960366 4/17/2016 12:00:00 AM                 1                700
## 6 1503960366 4/19/2016 12:00:00 AM                 1                304
##   TotalTimeInBed
## 1            346
## 2            407
## 3            442
## 4            367
## 5            712
## 6            320
\end{verbatim}

\begin{Shaded}
\begin{Highlighting}[]
\FunctionTok{head}\NormalTok{(steps)}
\end{Highlighting}
\end{Shaded}

\begin{verbatim}
##           Id ActivityDay StepTotal
## 1 1503960366   4/12/2016     13162
## 2 1503960366   4/13/2016     10735
## 3 1503960366   4/14/2016     10460
## 4 1503960366   4/15/2016      9762
## 5 1503960366   4/16/2016     12669
## 6 1503960366   4/17/2016      9705
\end{verbatim}

\begin{Shaded}
\begin{Highlighting}[]
\FunctionTok{head}\NormalTok{(weight)}
\end{Highlighting}
\end{Shaded}

\begin{verbatim}
##           Id                  Date WeightKg WeightPounds Fat   BMI
## 1 1503960366  5/2/2016 11:59:59 PM     52.6     115.9631  22 22.65
## 2 1503960366  5/3/2016 11:59:59 PM     52.6     115.9631  NA 22.65
## 3 1927972279  4/13/2016 1:08:52 AM    133.5     294.3171  NA 47.54
## 4 2873212765 4/21/2016 11:59:59 PM     56.7     125.0021  NA 21.45
## 5 2873212765 5/12/2016 11:59:59 PM     57.3     126.3249  NA 21.69
## 6 4319703577 4/17/2016 11:59:59 PM     72.4     159.6147  25 27.45
##   IsManualReport        LogId
## 1           True 1.462234e+12
## 2           True 1.462320e+12
## 3          False 1.460510e+12
## 4           True 1.461283e+12
## 5           True 1.463098e+12
## 6           True 1.460938e+12
\end{verbatim}

\subsubsection{Before starting we take a deep look at our summary
statistics especially MAX MIN, to fully undstand and be able to avoid
any errors while working with
data.}\label{before-starting-we-take-a-deep-look-at-our-summary-statistics-especially-max-min-to-fully-undstand-and-be-able-to-avoid-any-errors-while-working-with-data.}

\begin{Shaded}
\begin{Highlighting}[]
\CommentTok{\# Understanding Our tables for Better Decision and Understanding Statistical Rates}

\FunctionTok{summary}\NormalTok{(activity)}
\end{Highlighting}
\end{Shaded}

\begin{verbatim}
##        Id            ActivityDate         TotalSteps    TotalDistance   
##  Min.   :1.504e+09   Length:940         Min.   :    0   Min.   : 0.000  
##  1st Qu.:2.320e+09   Class :character   1st Qu.: 3790   1st Qu.: 2.620  
##  Median :4.445e+09   Mode  :character   Median : 7406   Median : 5.245  
##  Mean   :4.855e+09                      Mean   : 7638   Mean   : 5.490  
##  3rd Qu.:6.962e+09                      3rd Qu.:10727   3rd Qu.: 7.713  
##  Max.   :8.878e+09                      Max.   :36019   Max.   :28.030  
##  TrackerDistance  LoggedActivitiesDistance VeryActiveDistance
##  Min.   : 0.000   Min.   :0.0000           Min.   : 0.000    
##  1st Qu.: 2.620   1st Qu.:0.0000           1st Qu.: 0.000    
##  Median : 5.245   Median :0.0000           Median : 0.210    
##  Mean   : 5.475   Mean   :0.1082           Mean   : 1.503    
##  3rd Qu.: 7.710   3rd Qu.:0.0000           3rd Qu.: 2.053    
##  Max.   :28.030   Max.   :4.9421           Max.   :21.920    
##  ModeratelyActiveDistance LightActiveDistance SedentaryActiveDistance
##  Min.   :0.0000           Min.   : 0.000      Min.   :0.000000       
##  1st Qu.:0.0000           1st Qu.: 1.945      1st Qu.:0.000000       
##  Median :0.2400           Median : 3.365      Median :0.000000       
##  Mean   :0.5675           Mean   : 3.341      Mean   :0.001606       
##  3rd Qu.:0.8000           3rd Qu.: 4.782      3rd Qu.:0.000000       
##  Max.   :6.4800           Max.   :10.710      Max.   :0.110000       
##  VeryActiveMinutes FairlyActiveMinutes LightlyActiveMinutes SedentaryMinutes
##  Min.   :  0.00    Min.   :  0.00      Min.   :  0.0        Min.   :   0.0  
##  1st Qu.:  0.00    1st Qu.:  0.00      1st Qu.:127.0        1st Qu.: 729.8  
##  Median :  4.00    Median :  6.00      Median :199.0        Median :1057.5  
##  Mean   : 21.16    Mean   : 13.56      Mean   :192.8        Mean   : 991.2  
##  3rd Qu.: 32.00    3rd Qu.: 19.00      3rd Qu.:264.0        3rd Qu.:1229.5  
##  Max.   :210.00    Max.   :143.00      Max.   :518.0        Max.   :1440.0  
##     Calories   
##  Min.   :   0  
##  1st Qu.:1828  
##  Median :2134  
##  Mean   :2304  
##  3rd Qu.:2793  
##  Max.   :4900
\end{verbatim}

\begin{Shaded}
\begin{Highlighting}[]
\FunctionTok{summary}\NormalTok{(calories)}
\end{Highlighting}
\end{Shaded}

\begin{verbatim}
##        Id            ActivityDay           Calories   
##  Min.   :1.504e+09   Length:940         Min.   :   0  
##  1st Qu.:2.320e+09   Class :character   1st Qu.:1828  
##  Median :4.445e+09   Mode  :character   Median :2134  
##  Mean   :4.855e+09                      Mean   :2304  
##  3rd Qu.:6.962e+09                      3rd Qu.:2793  
##  Max.   :8.878e+09                      Max.   :4900
\end{verbatim}

\begin{Shaded}
\begin{Highlighting}[]
\FunctionTok{summary}\NormalTok{(dsleep)}
\end{Highlighting}
\end{Shaded}

\begin{verbatim}
##        Id              SleepDay         TotalSleepRecords TotalMinutesAsleep
##  Min.   :1.504e+09   Length:413         Min.   :1.000     Min.   : 58.0     
##  1st Qu.:3.977e+09   Class :character   1st Qu.:1.000     1st Qu.:361.0     
##  Median :4.703e+09   Mode  :character   Median :1.000     Median :433.0     
##  Mean   :5.001e+09                      Mean   :1.119     Mean   :419.5     
##  3rd Qu.:6.962e+09                      3rd Qu.:1.000     3rd Qu.:490.0     
##  Max.   :8.792e+09                      Max.   :3.000     Max.   :796.0     
##  TotalTimeInBed 
##  Min.   : 61.0  
##  1st Qu.:403.0  
##  Median :463.0  
##  Mean   :458.6  
##  3rd Qu.:526.0  
##  Max.   :961.0
\end{verbatim}

\begin{Shaded}
\begin{Highlighting}[]
\FunctionTok{summary}\NormalTok{(steps)}
\end{Highlighting}
\end{Shaded}

\begin{verbatim}
##        Id            ActivityDay          StepTotal    
##  Min.   :1.504e+09   Length:940         Min.   :    0  
##  1st Qu.:2.320e+09   Class :character   1st Qu.: 3790  
##  Median :4.445e+09   Mode  :character   Median : 7406  
##  Mean   :4.855e+09                      Mean   : 7638  
##  3rd Qu.:6.962e+09                      3rd Qu.:10727  
##  Max.   :8.878e+09                      Max.   :36019
\end{verbatim}

\begin{Shaded}
\begin{Highlighting}[]
\FunctionTok{summary}\NormalTok{(weight)}
\end{Highlighting}
\end{Shaded}

\begin{verbatim}
##        Id                Date              WeightKg       WeightPounds  
##  Min.   :1.504e+09   Length:67          Min.   : 52.60   Min.   :116.0  
##  1st Qu.:6.962e+09   Class :character   1st Qu.: 61.40   1st Qu.:135.4  
##  Median :6.962e+09   Mode  :character   Median : 62.50   Median :137.8  
##  Mean   :7.009e+09                      Mean   : 72.04   Mean   :158.8  
##  3rd Qu.:8.878e+09                      3rd Qu.: 85.05   3rd Qu.:187.5  
##  Max.   :8.878e+09                      Max.   :133.50   Max.   :294.3  
##                                                                         
##       Fat             BMI        IsManualReport         LogId          
##  Min.   :22.00   Min.   :21.45   Length:67          Min.   :1.460e+12  
##  1st Qu.:22.75   1st Qu.:23.96   Class :character   1st Qu.:1.461e+12  
##  Median :23.50   Median :24.39   Mode  :character   Median :1.462e+12  
##  Mean   :23.50   Mean   :25.19                      Mean   :1.462e+12  
##  3rd Qu.:24.25   3rd Qu.:25.56                      3rd Qu.:1.462e+12  
##  Max.   :25.00   Max.   :47.54                      Max.   :1.463e+12  
##  NA's   :65
\end{verbatim}

\section{4. Exploratory Data Analysis
(EDA)}\label{exploratory-data-analysis-eda}

\subsubsection{First we notice that date in most tables are considered
strings for further data integrity, we need to convert it first for
appropiate
use.}\label{first-we-notice-that-date-in-most-tables-are-considered-strings-for-further-data-integrity-we-need-to-convert-it-first-for-appropiate-use.}

\begin{Shaded}
\begin{Highlighting}[]
\CommentTok{\# Preparing to merge and work with final Data}

\NormalTok{activity }\OtherTok{\textless{}{-}}\NormalTok{ activity }\SpecialCharTok{\%\textgreater{}\%} 
  \FunctionTok{rename}\NormalTok{(}\AttributeTok{date =}\NormalTok{ ActivityDate) }\SpecialCharTok{\%\textgreater{}\%} 
  \FunctionTok{mutate}\NormalTok{(}\AttributeTok{date =} \FunctionTok{as.Date}\NormalTok{(date, }\AttributeTok{format =} \StringTok{"\%m/\%d/\%y"}\NormalTok{))}
\NormalTok{calories }\OtherTok{\textless{}{-}}\NormalTok{ calories }\SpecialCharTok{\%\textgreater{}\%} 
  \FunctionTok{rename}\NormalTok{(}\AttributeTok{date =}\NormalTok{ ActivityDay) }\SpecialCharTok{\%\textgreater{}\%} 
  \FunctionTok{mutate}\NormalTok{(}\AttributeTok{date =} \FunctionTok{as.Date}\NormalTok{(date, }\AttributeTok{format =} \StringTok{"\%m/\%d/\%y"}\NormalTok{))}
\NormalTok{dsleep }\OtherTok{\textless{}{-}}\NormalTok{ dsleep }\SpecialCharTok{\%\textgreater{}\%} 
  \FunctionTok{rename}\NormalTok{(}\AttributeTok{date =}\NormalTok{ SleepDay) }\SpecialCharTok{\%\textgreater{}\%} 
  \FunctionTok{mutate}\NormalTok{(}\AttributeTok{date =} \FunctionTok{as.Date}\NormalTok{(date, }\AttributeTok{format =} \StringTok{"\%m/\%d/\%y"}\NormalTok{))}
\NormalTok{steps }\OtherTok{\textless{}{-}}\NormalTok{ steps }\SpecialCharTok{\%\textgreater{}\%} 
  \FunctionTok{rename}\NormalTok{( }\AttributeTok{date=}\NormalTok{ ActivityDay) }\SpecialCharTok{\%\textgreater{}\%} 
  \FunctionTok{mutate}\NormalTok{(}\AttributeTok{date =} \FunctionTok{as.Date}\NormalTok{(date, }\AttributeTok{format =} \StringTok{"\%m/\%d/\%y"}\NormalTok{))}
\NormalTok{weight }\OtherTok{\textless{}{-}}\NormalTok{ weight }\SpecialCharTok{\%\textgreater{}\%} 
  \FunctionTok{rename}\NormalTok{(}\AttributeTok{date =}\NormalTok{ Date) }\SpecialCharTok{\%\textgreater{}\%} 
  \FunctionTok{mutate}\NormalTok{(}\AttributeTok{date =} \FunctionTok{as.Date}\NormalTok{(date, }\AttributeTok{format =} \StringTok{"\%m/\%d/\%y"}\NormalTok{))}
\NormalTok{hcalories }\OtherTok{\textless{}{-}}\NormalTok{ hcalories }\SpecialCharTok{\%\textgreater{}\%} 
  \FunctionTok{rename}\NormalTok{(}\AttributeTok{date =}\NormalTok{ ActivityHour) }\SpecialCharTok{\%\textgreater{}\%} 
  \FunctionTok{mutate}\NormalTok{(}\AttributeTok{date =} \FunctionTok{as.Date}\NormalTok{(date, }\AttributeTok{format =} \StringTok{"\%m/\%d/\%y"}\NormalTok{))}
\end{Highlighting}
\end{Shaded}

\subsubsection{Finally After Checking all data and preprocessing some
data key elements, we begin to create our final table to start working
with.}\label{finally-after-checking-all-data-and-preprocessing-some-data-key-elements-we-begin-to-create-our-final-table-to-start-working-with.}

\begin{Shaded}
\begin{Highlighting}[]
\CommentTok{\#Memory allocation would be failure to join without Adjusting date}

\NormalTok{finaltable }\OtherTok{\textless{}{-}}\NormalTok{ activity }\SpecialCharTok{\%\textgreater{}\%}
  \FunctionTok{left\_join}\NormalTok{(hcalories, }\AttributeTok{by =} \FunctionTok{c}\NormalTok{(}\StringTok{"Id"}\NormalTok{,}\StringTok{"date"}\NormalTok{)) }\SpecialCharTok{\%\textgreater{}\%}
  \FunctionTok{left\_join}\NormalTok{(calories, }\AttributeTok{by =} \FunctionTok{c}\NormalTok{(}\StringTok{"Id"}\NormalTok{,}\StringTok{"date"}\NormalTok{)) }\SpecialCharTok{\%\textgreater{}\%}
  \FunctionTok{left\_join}\NormalTok{(steps, }\AttributeTok{by =} \FunctionTok{c}\NormalTok{(}\StringTok{"Id"}\NormalTok{,}\StringTok{"date"}\NormalTok{)) }\SpecialCharTok{\%\textgreater{}\%}
  \FunctionTok{left\_join}\NormalTok{(weight, }\AttributeTok{by =} \FunctionTok{c}\NormalTok{(}\StringTok{"Id"}\NormalTok{,}\StringTok{"date"}\NormalTok{)) }\SpecialCharTok{\%\textgreater{}\%}
  \FunctionTok{left\_join}\NormalTok{(dsleep, }\AttributeTok{by =} \FunctionTok{c}\NormalTok{(}\StringTok{"Id"}\NormalTok{,}\StringTok{"date"}\NormalTok{), }\AttributeTok{relationship =} \StringTok{"many{-}to{-}many"}\NormalTok{)}
\end{Highlighting}
\end{Shaded}

\subsubsection{For better view of data its better to remove unecessary
columns}\label{for-better-view-of-data-its-better-to-remove-unecessary-columns}

\subsubsection{Viewing finaltable and importing it for further
usage.}\label{viewing-finaltable-and-importing-it-for-further-usage.}

\begin{Shaded}
\begin{Highlighting}[]
\CommentTok{\#Final Clean Form to Work With}



\CommentTok{\#Exporting Data Set for Clean work with Visualization}
\FunctionTok{write\_csv}\NormalTok{(finaltable, }\StringTok{"FinalWorkingTable.csv"}\NormalTok{)  }\CommentTok{\# row.names = FALSE}
\end{Highlighting}
\end{Shaded}

\section{5. Visualizations}\label{visualizations}

\subsubsection{First we plot the relation between distance and burning
calories to show when exactly burning calories efficiency taking place,
and understanding how important is to monitor
distance.}\label{first-we-plot-the-relation-between-distance-and-burning-calories-to-show-when-exactly-burning-calories-efficiency-taking-place-and-understanding-how-important-is-to-monitor-distance.}

\begin{Shaded}
\begin{Highlighting}[]
\CommentTok{\#Visualization Relationship}

\CommentTok{\#Undstanding Distance to Burning Calories}

\NormalTok{T1 }\OtherTok{\textless{}{-}}  \FunctionTok{ggplot}\NormalTok{(finaltable, }\FunctionTok{aes}\NormalTok{(}\AttributeTok{x=}\NormalTok{VeryActiveDistance, }\AttributeTok{y=}\NormalTok{Calories)) }\SpecialCharTok{+} \FunctionTok{geom\_point}\NormalTok{(}\FunctionTok{aes}\NormalTok{(}\AttributeTok{color=}\NormalTok{TotalSteps)) }\SpecialCharTok{+} \FunctionTok{geom\_smooth}\NormalTok{() }\SpecialCharTok{+} \FunctionTok{labs}\NormalTok{(}\AttributeTok{x=}\StringTok{"Active Distance Covered"}\NormalTok{, }\AttributeTok{y=}\StringTok{"Calories Burned"}\NormalTok{)}
\NormalTok{T2 }\OtherTok{\textless{}{-}}  \FunctionTok{ggplot}\NormalTok{(finaltable, }\FunctionTok{aes}\NormalTok{(}\AttributeTok{x=}\NormalTok{ModeratelyActiveDistance, }\AttributeTok{y=}\NormalTok{Calories)) }\SpecialCharTok{+} \FunctionTok{geom\_point}\NormalTok{(}\FunctionTok{aes}\NormalTok{(}\AttributeTok{color=}\NormalTok{TotalSteps)) }\SpecialCharTok{+} \FunctionTok{geom\_smooth}\NormalTok{() }\SpecialCharTok{+} \FunctionTok{labs}\NormalTok{(}\AttributeTok{x=}\StringTok{"Moderate Activity Distance Covered"}\NormalTok{, }\AttributeTok{y=}\StringTok{"Calories Burned"}\NormalTok{)}
\NormalTok{T3 }\OtherTok{\textless{}{-}}  \FunctionTok{ggplot}\NormalTok{(finaltable, }\FunctionTok{aes}\NormalTok{(}\AttributeTok{x=}\NormalTok{LightActiveDistance, }\AttributeTok{y=}\NormalTok{Calories)) }\SpecialCharTok{+} \FunctionTok{geom\_point}\NormalTok{(}\FunctionTok{aes}\NormalTok{(}\AttributeTok{color=}\NormalTok{TotalSteps)) }\SpecialCharTok{+} \FunctionTok{geom\_smooth}\NormalTok{() }\SpecialCharTok{+} \FunctionTok{labs}\NormalTok{(}\AttributeTok{x=}\StringTok{"Light Activity Distance Covered"}\NormalTok{, }\AttributeTok{y=}\StringTok{"Calories Burned"}\NormalTok{)}
\NormalTok{T4 }\OtherTok{\textless{}{-}}  \FunctionTok{ggplot}\NormalTok{(finaltable, }\FunctionTok{aes}\NormalTok{(}\AttributeTok{x=}\NormalTok{SedentaryActiveDistance, }\AttributeTok{y=}\NormalTok{Calories)) }\SpecialCharTok{+} \FunctionTok{geom\_point}\NormalTok{(}\FunctionTok{aes}\NormalTok{(}\AttributeTok{color=}\NormalTok{TotalSteps)) }\SpecialCharTok{+} \FunctionTok{geom\_smooth}\NormalTok{() }\SpecialCharTok{+} \FunctionTok{labs}\NormalTok{(}\AttributeTok{x=}\StringTok{"Sedentary Activity Distance Covered"}\NormalTok{, }\AttributeTok{y=}\StringTok{"Calories Burned"}\NormalTok{)}

\FunctionTok{grid.arrange}\NormalTok{(T1,T2,T3,T4)}
\end{Highlighting}
\end{Shaded}

\begin{verbatim}
## `geom_smooth()` using method = 'gam' and formula = 'y ~ s(x, bs = "cs")'
## `geom_smooth()` using method = 'gam' and formula = 'y ~ s(x, bs = "cs")'
## `geom_smooth()` using method = 'gam' and formula = 'y ~ s(x, bs = "cs")'
## `geom_smooth()` using method = 'gam' and formula = 'y ~ s(x, bs = "cs")'
\end{verbatim}

\begin{verbatim}
## Warning: Failed to fit group -1.
## Caused by error in `smooth.construct.cr.smooth.spec()`:
## ! x has insufficient unique values to support 10 knots: reduce k.
\end{verbatim}

\includegraphics{Final-Report_files/figure-latex/unnamed-chunk-1-1.pdf}

\subsubsection{Then we plot the relation between Time and burning
calories to show how both distance and time effect equally burning
calories and how important both parameters, and understanding how
important is to monitor also time along side
distance.}\label{then-we-plot-the-relation-between-time-and-burning-calories-to-show-how-both-distance-and-time-effect-equally-burning-calories-and-how-important-both-parameters-and-understanding-how-important-is-to-monitor-also-time-along-side-distance.}

\begin{Shaded}
\begin{Highlighting}[]
\CommentTok{\#Understanding Time used for Burning Calories}


\NormalTok{M1 }\OtherTok{\textless{}{-}} \FunctionTok{ggplot}\NormalTok{(finaltable, }\FunctionTok{aes}\NormalTok{(}\AttributeTok{x=}\NormalTok{VeryActiveMinutes, }\AttributeTok{y=}\NormalTok{Calories)) }\SpecialCharTok{+} \FunctionTok{geom\_point}\NormalTok{(}\FunctionTok{aes}\NormalTok{(}\AttributeTok{color=}\NormalTok{VeryActiveMinutes)) }\SpecialCharTok{+} \FunctionTok{geom\_smooth}\NormalTok{() }\SpecialCharTok{+} \FunctionTok{labs}\NormalTok{(}\AttributeTok{x=}\StringTok{"Active Minutes Consumed"}\NormalTok{, }\AttributeTok{y=}\StringTok{"Calories Burned"}\NormalTok{,}\AttributeTok{color=}\StringTok{"Minutes"}\NormalTok{)}
\NormalTok{M2 }\OtherTok{\textless{}{-}} \FunctionTok{ggplot}\NormalTok{(finaltable, }\FunctionTok{aes}\NormalTok{(}\AttributeTok{x=}\NormalTok{FairlyActiveMinutes, }\AttributeTok{y=}\NormalTok{Calories)) }\SpecialCharTok{+} \FunctionTok{geom\_point}\NormalTok{(}\FunctionTok{aes}\NormalTok{(}\AttributeTok{color=}\NormalTok{FairlyActiveMinutes)) }\SpecialCharTok{+} \FunctionTok{geom\_smooth}\NormalTok{() }\SpecialCharTok{+} \FunctionTok{labs}\NormalTok{(}\AttributeTok{x=}\StringTok{"Fairly Active Minutes Consumed"}\NormalTok{, }\AttributeTok{y=}\StringTok{"Calories Burned"}\NormalTok{,}\AttributeTok{color=}\StringTok{"Minutes"}\NormalTok{)}
\NormalTok{M3 }\OtherTok{\textless{}{-}} \FunctionTok{ggplot}\NormalTok{(finaltable, }\FunctionTok{aes}\NormalTok{(}\AttributeTok{x=}\NormalTok{LightlyActiveMinutes, }\AttributeTok{y=}\NormalTok{Calories)) }\SpecialCharTok{+} \FunctionTok{geom\_point}\NormalTok{(}\FunctionTok{aes}\NormalTok{(}\AttributeTok{color=}\NormalTok{LightlyActiveMinutes)) }\SpecialCharTok{+} \FunctionTok{geom\_smooth}\NormalTok{() }\SpecialCharTok{+} \FunctionTok{labs}\NormalTok{(}\AttributeTok{x=}\StringTok{"Lightly Active Minutes Consumed"}\NormalTok{, }\AttributeTok{y=}\StringTok{"Calories Burned"}\NormalTok{,}\AttributeTok{color=}\StringTok{"Minutes"}\NormalTok{)}
\NormalTok{M4 }\OtherTok{\textless{}{-}} \FunctionTok{ggplot}\NormalTok{(finaltable, }\FunctionTok{aes}\NormalTok{(}\AttributeTok{x=}\NormalTok{SedentaryMinutes, }\AttributeTok{y=}\NormalTok{Calories)) }\SpecialCharTok{+} \FunctionTok{geom\_point}\NormalTok{(}\FunctionTok{aes}\NormalTok{(}\AttributeTok{color=}\NormalTok{SedentaryMinutes)) }\SpecialCharTok{+} \FunctionTok{geom\_smooth}\NormalTok{() }\SpecialCharTok{+} \FunctionTok{labs}\NormalTok{(}\AttributeTok{x=}\StringTok{"Sedentary Minutes Consumed"}\NormalTok{, }\AttributeTok{y=}\StringTok{"Calories Burned"}\NormalTok{,}\AttributeTok{color=}\StringTok{"Minutes"}\NormalTok{)}

\FunctionTok{grid.arrange}\NormalTok{(M1,M2,M3,M4)}
\end{Highlighting}
\end{Shaded}

\begin{verbatim}
## `geom_smooth()` using method = 'gam' and formula = 'y ~ s(x, bs = "cs")'
## `geom_smooth()` using method = 'gam' and formula = 'y ~ s(x, bs = "cs")'
## `geom_smooth()` using method = 'gam' and formula = 'y ~ s(x, bs = "cs")'
## `geom_smooth()` using method = 'gam' and formula = 'y ~ s(x, bs = "cs")'
\end{verbatim}

\includegraphics{Final-Report_files/figure-latex/unnamed-chunk-2-1.pdf}

\subsubsection{After successfully showing how important is to burn
calories and efficiency of monitoring both distance and time, we begin
further more investigating Other parameters to show case the usefulness
and the need of monitoring all parameters at
once.}\label{after-successfully-showing-how-important-is-to-burn-calories-and-efficiency-of-monitoring-both-distance-and-time-we-begin-further-more-investigating-other-parameters-to-show-case-the-usefulness-and-the-need-of-monitoring-all-parameters-at-once.}

\begin{Shaded}
\begin{Highlighting}[]
\CommentTok{\# Understanding Affected Parameters to Burning Calories}

\NormalTok{Tsteps }\OtherTok{\textless{}{-}} \FunctionTok{ggplot}\NormalTok{(finaltable, }\FunctionTok{aes}\NormalTok{(}\AttributeTok{x=}\NormalTok{TotalSteps, }\AttributeTok{y=}\NormalTok{Calories)) }\SpecialCharTok{+} \FunctionTok{geom\_point}\NormalTok{(}\FunctionTok{aes}\NormalTok{(}\AttributeTok{color=}\NormalTok{TotalSteps)) }\SpecialCharTok{+} \FunctionTok{geom\_smooth}\NormalTok{() }\SpecialCharTok{+} \FunctionTok{labs}\NormalTok{(}\AttributeTok{x=}\StringTok{"Steps"}\NormalTok{, }\AttributeTok{y=}\StringTok{"Calories Burned"}\NormalTok{,}\AttributeTok{color=}\StringTok{"Steps"}\NormalTok{)}
\NormalTok{TDistance }\OtherTok{\textless{}{-}} \FunctionTok{ggplot}\NormalTok{(finaltable, }\FunctionTok{aes}\NormalTok{(}\AttributeTok{x=}\NormalTok{TotalDistance, }\AttributeTok{y=}\NormalTok{Calories)) }\SpecialCharTok{+} \FunctionTok{geom\_point}\NormalTok{(}\FunctionTok{aes}\NormalTok{(}\AttributeTok{color=}\NormalTok{TotalDistance)) }\SpecialCharTok{+} \FunctionTok{geom\_smooth}\NormalTok{() }\SpecialCharTok{+} \FunctionTok{labs}\NormalTok{(}\AttributeTok{x=}\StringTok{"Total Distance"}\NormalTok{, }\AttributeTok{y=}\StringTok{"Calories Burned"}\NormalTok{,}\AttributeTok{color=}\StringTok{"Distance"}\NormalTok{)}
\NormalTok{TMS }\OtherTok{\textless{}{-}} \FunctionTok{ggplot}\NormalTok{(finaltable, }\FunctionTok{aes}\NormalTok{(}\AttributeTok{x=}\NormalTok{TotalMinutesAsleep, }\AttributeTok{y=}\NormalTok{Calories)) }\SpecialCharTok{+} \FunctionTok{geom\_point}\NormalTok{(}\FunctionTok{aes}\NormalTok{(}\AttributeTok{color=}\NormalTok{TotalMinutesAsleep)) }\SpecialCharTok{+} \FunctionTok{geom\_smooth}\NormalTok{() }\SpecialCharTok{+} \FunctionTok{labs}\NormalTok{(}\AttributeTok{x=}\StringTok{"Total Sleep in Minutes"}\NormalTok{, }\AttributeTok{y=}\StringTok{"Calories Burned"}\NormalTok{,}\AttributeTok{color=}\StringTok{"Total Sleeping in Minutes"}\NormalTok{)}
\NormalTok{TTB }\OtherTok{\textless{}{-}} \FunctionTok{ggplot}\NormalTok{(finaltable, }\FunctionTok{aes}\NormalTok{(}\AttributeTok{x=}\NormalTok{TotalTimeInBed, }\AttributeTok{y=}\NormalTok{Calories)) }\SpecialCharTok{+} \FunctionTok{geom\_point}\NormalTok{(}\FunctionTok{aes}\NormalTok{(}\AttributeTok{color=}\NormalTok{TotalTimeInBed)) }\SpecialCharTok{+} \FunctionTok{geom\_smooth}\NormalTok{() }\SpecialCharTok{+} \FunctionTok{labs}\NormalTok{(}\AttributeTok{x=}\StringTok{"Total Time in Bed"}\NormalTok{, }\AttributeTok{y=}\StringTok{"Calories Burned"}\NormalTok{,}\AttributeTok{color=}\StringTok{"Total Time In Bed"}\NormalTok{)}
\NormalTok{WC }\OtherTok{\textless{}{-}} \FunctionTok{ggplot}\NormalTok{(finaltable, }\FunctionTok{aes}\NormalTok{(}\AttributeTok{x=}\NormalTok{WeightKg, }\AttributeTok{y=}\NormalTok{Calories)) }\SpecialCharTok{+} \FunctionTok{geom\_point}\NormalTok{(}\FunctionTok{aes}\NormalTok{(}\AttributeTok{color=}\NormalTok{WeightKg)) }\SpecialCharTok{+} \FunctionTok{geom\_smooth}\NormalTok{() }\SpecialCharTok{+} \FunctionTok{labs}\NormalTok{(}\AttributeTok{x=}\StringTok{"Weight"}\NormalTok{, }\AttributeTok{y=}\StringTok{"Calories Burned"}\NormalTok{,}\AttributeTok{color=}\StringTok{"Weight"}\NormalTok{)}
\NormalTok{TTZ }\OtherTok{\textless{}{-}} \FunctionTok{ggplot}\NormalTok{(finaltable, }\FunctionTok{aes}\NormalTok{(}\AttributeTok{x=}\NormalTok{TotalTimeInBed, }\AttributeTok{y=}\NormalTok{TotalMinutesAsleep)) }\SpecialCharTok{+} \FunctionTok{geom\_step}\NormalTok{(}\AttributeTok{direction=}\StringTok{"hv"}\NormalTok{) }\SpecialCharTok{+} \FunctionTok{geom\_smooth}\NormalTok{() }\SpecialCharTok{+} \FunctionTok{labs}\NormalTok{(}\AttributeTok{x=}\StringTok{"Total Time in Bed"}\NormalTok{, }\AttributeTok{y=}\StringTok{"Total Time A Sleep"}\NormalTok{,}\AttributeTok{color=}\StringTok{"Difference Between Sleep and Spended time in Bed"}\NormalTok{)}
\FunctionTok{grid.arrange}\NormalTok{(Tsteps,TDistance,TMS,WC,TTB,TTZ)}
\end{Highlighting}
\end{Shaded}

\begin{verbatim}
## `geom_smooth()` using method = 'gam' and formula = 'y ~ s(x, bs = "cs")'
## `geom_smooth()` using method = 'gam' and formula = 'y ~ s(x, bs = "cs")'
## `geom_smooth()` using method = 'gam' and formula = 'y ~ s(x, bs = "cs")'
\end{verbatim}

\begin{verbatim}
## Warning: Removed 12406 rows containing non-finite outside the scale range
## (`stat_smooth()`).
\end{verbatim}

\begin{verbatim}
## Warning: Removed 12406 rows containing missing values or values outside the scale range
## (`geom_point()`).
\end{verbatim}

\begin{verbatim}
## `geom_smooth()` using method = 'gam' and formula = 'y ~ s(x, bs = "cs")'
\end{verbatim}

\begin{verbatim}
## Warning: Removed 20598 rows containing non-finite outside the scale range
## (`stat_smooth()`).
\end{verbatim}

\begin{verbatim}
## Warning: Removed 20598 rows containing missing values or values outside the scale range
## (`geom_point()`).
\end{verbatim}

\begin{verbatim}
## `geom_smooth()` using method = 'gam' and formula = 'y ~ s(x, bs = "cs")'
\end{verbatim}

\begin{verbatim}
## Warning: Removed 12406 rows containing non-finite outside the scale range
## (`stat_smooth()`).
\end{verbatim}

\begin{verbatim}
## Warning: Removed 12406 rows containing missing values or values outside the scale range
## (`geom_point()`).
\end{verbatim}

\begin{verbatim}
## `geom_smooth()` using method = 'gam' and formula = 'y ~ s(x, bs = "cs")'
\end{verbatim}

\begin{verbatim}
## Warning: Removed 12406 rows containing non-finite outside the scale range
## (`stat_smooth()`).
\end{verbatim}

\begin{verbatim}
## Warning: Removed 856 rows containing missing values or values outside the scale range
## (`geom_step()`).
\end{verbatim}

\includegraphics{Final-Report_files/figure-latex/unnamed-chunk-3-1.pdf}

\section{6. Act Phase.}\label{act-phase.}

\subsubsection{Upon further investigation needed for more detailed and
deceision
making,}\label{upon-further-investigation-needed-for-more-detailed-and-deceision-making}

\subsubsection{Though from previous analysis we understanding how
important to monitor multiple
parameters.}\label{though-from-previous-analysis-we-understanding-how-important-to-monitor-multiple-parameters.}

\subsubsection{For efficient calorie burning and to relate the following
studies to provide how important, for health is calorie management and
monitoring for maxmize health
benefits.}\label{for-efficient-calorie-burning-and-to-relate-the-following-studies-to-provide-how-important-for-health-is-calorie-management-and-monitoring-for-maxmize-health-benefits.}

\paragraph{A book to further investigate health concerns and how calorie
management can eliminate Risk of Chronic and
diseases.}\label{a-book-to-further-investigate-health-concerns-and-how-calorie-management-can-eliminate-risk-of-chronic-and-diseases.}

\href{https://www.ncbi.nlm.nih.gov/books/NBK235013/}{Eat for Life: The
Food and Nutrition Board's Guide to Reducing Your Risk of Chronic
Disease. by American NIH}

\paragraph{An NIH Research Article matters how important for organize
and benefits of calorie
management.}\label{an-nih-research-article-matters-how-important-for-organize-and-benefits-of-calorie-management.}

\href{https://www.nih.gov/news-events/nih-research-matters/calorie-restriction-immune-function-health-span}{Calorie
restriction, immune function, and health span}

\end{document}
